\documentclass{elsart}
\usepackage[numbers]{natbib}
\newcommand{\threej}[6]{\ensuremath{\left({#1\atop #4}{#2\atop #5}
{#3\atop #6}\right)}}
\newcommand{\sixj}[6]{\ensuremath{\left\{{#1\atop #4}{#2\atop #5}
{#3\atop #6}\right\}}}

\newcommand{\aaa}{Astron. Astrophys.}
\newcommand{\pra}{Phys. Rev. A}
\newcommand{\cpc}{Comput. Phys. Commun.}
\newcommand{\jpb}{J. Phys. B}
\newcommand{\ant}{At. Data Nucl. Data Tables}
\newcommand{\amp}{Adv. Atom. Mol. Phys.}
\newcommand{\cc}{Astron. Astrophys.}
\newcommand{\apjs}{ApJ. Suppl.}

\defcitealias{gu01a}{Paper~I}
\defcitealias{gu01b}{Paper~II}

\begin{document}

\begin{frontmatter}

\title{Flexible Atomic Code: III. Photoionization and Radiative Recombination} 
\author{Ming Feng Gu}
\address{Massachusetts Institute of Technology, Cambridge, MA 02139, USA}

\begin{abstract} 
A computer program for the calculation of non-resonant photoionization and
radiative recombination cross sections is presented. The continuum
wavefunctions are obtained in the relativistic distorted-wave
approximation. The non-relativistic multipole operators are used for the
calculation of bound-free differential oscillator strengths, which should be
accurate for photon energies less than the electron rest mass. An efficient
factorization-interpolation procedure is implemented by separating the
coupling of the continuum electron from the bound states.
\end{abstract}

\begin{keyword}
Distorted-wave \sep Photoionization \sep Radiative recombination. 
\PACS 32.80.Fb
\end{keyword}
\end{frontmatter}

\textbf{\large PROGRAM SUMMARY}

Title of program: FAC

Catalogue identifier:

Distribution formate:

Computer for which the program is designed and others on which the program has
been tested:\\
Computers: Sun Workstations, PCs\\
Operating systems under which the program has been tested: Unix

Programming language used: C, Fortran, Python

Number of bits in a word: 64

Number of processors used: 1

Has the code been vectorized or parallelized: No

Keywords: Distorted-wave, Photoionization, Radiative recombination.

Nature of physical problem: Photoionization and radiative recombination are
important processes determining the ionization balance and level population in
hot plasma. The cross sections for these processes must be known for reliable
interpretation of spectroscopical data in laboratory experiments and
astrophysics. 

Method of solution: The continuum wavefunctions are obtained in relativistic
distorted wave approximation. The bound-free differential oscillator strengths
are calculated using non-relativistic multipole operators, which is valid for
photon energies less than the electron rest mass. An efficient interpolation
procedure is used to calculate the radial integrals at many collision
energies. 

Restrictions on the complexity of program: No special restrictions except for
that of system resources. 

Typical running time: The test run takes 0.4 seconds on an Ultra-10 Sun
Workstation. 

\textbf{\large LONG WRITE-UP}

\section{Introduction}
This is the third of a sequence of papers describing a complete software
package for calculating various atomic processes, the Flexible Atomic Code
(FAC). The present paper describes a procedure for the calculation of
non-resonant photoionization (PI) and radiative recombination (RR) cross
sections in the distorted-wave (DW) approximation. Photoionization 
and radiative recombination are important processes that determine the
ionization balance of hot plasmas in variety of physical conditions. Knowledge
of photoionization cross sections is also essential for the calculation of
plasma opacity. In some circumstances, radiative recombination may be the
dominant process that populates the levels responsible for line emission. 

The most accurate methods for calculating low energy PI and
recombination (both RR and dielectronic recombination) cross sections are ones
based on the close-coupling (CC) approximation. The most widely used
implementation of the CC approximation is the 
R-matrix method \citep{hummer93, berrington95}. In this method, resonances are
treated in the most natural way through the coupling to closed channels. The
main drawback of the CC approximation is the prohibitively long computing time
required, especially at high energies \citep{zhang98}. The other class of
methods is based on the DW approximation, and ignores the inter-channel
coupling. The non-resonant background and the resonances are calculated
separately. In the present paper, we only discuss the non-resonant
PI and RR. Many calculations have been carried out using the DW approximation
employing different approaches to the bound and continuum wavefunctions
\citep{reilman79, clark86, verner93}. Most of previous theoretical results are
for PI from or RR onto a specific non-relativistic subshell, not for the
specific atomic states. \citet{zhang98} has extended their fully relativistic
code originally developed for the electron impact excitation (EIE) to the
calculation of PI and RR, and provided high energy PI cross sections to
complement the Opacity Project database \citep{seaton87}. The
present code is similar in technique to the program of \citet{zhang98}. To
improve the efficiency, we have extended the factorization-interpolation
procedure of \citet{barshalom88} first developed for the calculation of EIE
cross sections to the calculation of PI and RR cross sections. 
In addition, we have implemented the option to calculate PI and RR cross
sections through multipole operators other than electric dipole (E1), although
no interference between different multipoles is taken into account. 

In \S\ref{sec_theory} we present the factorized formalism of non-resonant PI 
and key numerical techniques of the computer program. \S\ref{sec_program}
introduces essential FAC routines for the calculation of PI and RR cross
sections. \S\ref{sec_example} illustrates the practical usage of these
routines through the calculation of the PI cross sections for Li-like
iron. The results are compared with those of \citet{zhang98}.

\section{Theory and Numerical Techniques}
\label{sec_theory}
\subsection{Factorized Formula for Cross Sections}
The partial PI cross section can be expressed in terms of the differential
oscillator strength (in atomic units), and the partial RR cross section is
related to that of PI through the Milne relation
\begin{eqnarray}
\sigma_{PI} &=& 2\pi^2\alpha\frac{\d f}{\d E} \nonumber\\
\sigma_{RR} &=& \frac{\alpha^2}{2}\frac{g_i}{g_f}\frac{\omega^2}{\varepsilon}
\sigma_{PI},
\end{eqnarray}
where $\alpha$ is the fine structure constant, $g_i$ and $g_f$ are the
statistical weight of the bound states before and after the photoionzation
takes place respectively, $\omega$ is the photon energy, and $\varepsilon$ is
the energy of the ejected photo-electron. 
The differential oscillator
strength, $\d f/\d E$, may be calculated similar to the bound-bound oscillator
strength through the generalized line strength
\begin{equation}
\frac{\d f}{\d E} = \frac{\omega}{g_i}[L]^{-1}(\alpha\omega)^{2L-2}S,
\end{equation}
where $L$ is the rank of the multipole operator inducing the transition,
$[L]$ denotes $2L+1$, and the generalized line strength is 
\begin{equation}
\label{eq_S}
S = \sum_{\kappa J_T}\left|<\psi_f,\kappa;J_T||O^L||\psi_i>\right|^2,
\end{equation}
where $\kappa$ is the relativistic quantum number of the free electron, $J_T$
is the total angular momentum of the free state when the final bound state is
coupled to the continuum electron, and $O^L$ is the multiple operator
inducing the transition.

As in the calculation of bound-bound radiative transitions,
Equation \ref{eq_S} may be decomposed into an angular part and a radial part
\citepalias{gu01a} 
\begin{equation}
\label{eq_Sf}
S = \sum_{\kappa J_T}\left|\sum_{\alpha\beta}
<\psi_f,\kappa;J_T||Z^L_M(\alpha,\beta)||\psi_i><\alpha||C^L||\beta>
M^L_{\alpha\beta}\right|^2,
\end{equation}
where $M^L_{\alpha\beta}$ is the one-electron radial integral and
$Z^L_M(\alpha,\beta)$ is the angular operator obtained by coupling the
creation and annihilation operators
\begin{equation}
Z^L_M(\alpha,\beta) = -[L]^{-1/2}\left[a^{\dagger}_{\hat{\alpha}}\times
\tilde{a}_{\hat{\beta}}\right],
\end{equation}
where the hat over the one electron state index indicates the inclusion of
magnetic quantum numbers $m$, and the tilde over the annihilation operator
denotes the transformation $\tilde{a}_{\hat{\beta}} = (-1)^{j_\beta -
m_\beta}a_{-\hat{\beta}}$ with $-\hat{\beta}$ being understood as the negation
of the magnetic quantum number $m_\beta$. 

Since the continuum orbital is absent in the initial bound state, the creation
operator $a^{\dagger}_{\tilde{\alpha}}$ must be chosen to be that of the free
electron. Using the decoupling formula of Racah, the reduced matrix elements
of the angular operator can be written as
\begin{eqnarray}
<\psi_f,\kappa;J_T||Z^L_M(\kappa,\beta)||\psi_i> &=&
 (-1)^{J_T+J_i-L}[J_T]^{1/2}
<\psi_f||\tilde{a}_{\tilde{\beta}}||\psi_i>\nonumber\\
&\times& \sixj{J_f}{j}{J_T}{L}{J_i}{j_\beta},
\end{eqnarray}
where $J_i$ and $J_f$ are the total angular momenta of the initial and final
states, $j$ is the angular momentum of the free electron, and
\sixj{j_1}{j_2}{j_3}{j_4}{j_5}{j_6} is the Wigner $6j$ symbol. Substituting
this into Equation \ref{eq_Sf} results in
\begin{eqnarray}
S &=&\sum_{\beta\beta^\prime}
<\psi_f||\tilde{a}_{\tilde{\beta}}||\psi_i>
<\psi_f||\tilde{a}_{\tilde{\beta^\prime}}||\psi_i>  
\nonumber\\
&\times& \sum_{\kappa}<\kappa||C^L||\beta><\kappa||C^L||\beta^\prime>
M^L_{\kappa\beta}M^L_{\kappa\beta^\prime}
\nonumber\\
&\times&\sum_{J_T}[J_T]\sixj{J_f}{j}{J_T}{L}{J_i}{j_\beta}
\sixj{J_f}{j}{J_T}{L}{J_i}{j_{\beta^\prime}}.
\end{eqnarray}
The summation over $J_T$ can be carried out analytically using the
orthogonality relation of $6j$ symbols to give
\begin{eqnarray}
S &=& \sum_{\beta\beta^\prime} \delta_{j_\beta j_{\beta^\prime}}[j_\beta]^{-1}
<\psi_f||\tilde{a}_{\tilde{\beta}}||\psi_i>
<\psi_f||\tilde{a}_{\tilde{\beta^\prime}}||\psi_i>  
\nonumber\\
&\times&\sum_{\kappa}<\kappa||C^L||\beta><\kappa||C^L||\beta^\prime>
M^L_{\kappa\beta}M^L_{\kappa\beta^\prime}.
\end{eqnarray}
Note that the condition $j_\beta = j_{\beta^\prime}$ in the summation also
implies $l_\beta = l_{\beta^\prime}$, since only basis states with the same
parity can mix in $\psi_i$. 

The initial and final bound states are expressed as a sum over basis states
$\Phi_\mu$. The angular coefficients
$<\psi_f||\tilde{a}_{\tilde{\beta}}||\psi_i>$ are calculated as
\begin{equation}
<\psi_f||\tilde{a}_{\hat{\beta}}||\psi_i> = \sum_{\mu\nu}b_\mu b_\nu
<\Phi_\mu||\tilde{a}_{\hat{\beta}}||\Phi_\nu>,
\end{equation}
where $b_\mu$ are the mixing coefficients. The reduced matrix elements
$<\Phi_\mu||\tilde{a}_{\tilde{\beta}}||\Phi_\nu>$, which are related to the
coefficients of fractional parentage, are calculated using the method of
\citet{gaigalas97}. 

\subsection{Construction of Recombined Bound States}
In the calculation of PI and RR cross sections, bound states in two
neighboring charge 
states are involved. The central potential entering the Dirac equation may be
constructed with either the configurations of the ion before photoionization
(recombined ion) or after photoionization (recombining ion). The difference
between the two choices is small for high $Z$ elements. However, it can be
substantial for low $Z$ elements. As discussed in \citetalias{gu01b}, the
effective nuclear charge experienced by the free electron at large radial
distances is screened by one additional electron as compared to that
experienced by the bound electrons in the
recombining ion, and is the same as that experienced by the bound electrons in
the recombined ion. This suggests that
one should use the mean configuration of the recombined ion for the
construction of the central potential. 

The Hamiltonian matrix for the recombined ion may be constructed as for the
recombining ion by specifying appropriate configurations. However, it is
sometimes beneficial to use an alternative approach, that utilizes the already
diagonalized Hamiltonian of the recombining ion. In this method, the basis
states are constructed by coupling one additional electron to a specific
atomic state 
of the recombining ion $|\psi>$. However, if $|\psi>$ has a shell which is
equivalent to the additional electron, the antisymmetrization requires
non-trivial recoupling. Such recoupling is not needed when the additional
electron is high lying and interacts with the core recombining state
weakly. That is, the added electrons can be treated as spectators. We shall use
the alternative approach only in this case. In addition, only 
mixing of the basis states that have the spectator electron in the same
complex is 
allowed. When more general configuration interaction is needed, one should
construct the recombined states the same way as for recombining ones.
Since the core states already diagonalize the Hamiltonian of the
recombining ion, the addition of a weakly interacting electron should not
introduce large mixing of basis functions with a common core state. In fact,
when the 
principal quantum number of the spectator electron is large, only the mixing
between basis functions with the same core state need to be included. This
reduces the dimension of the Hamiltonian matrices substantially. 

The evaluation of the Hamiltonian matrix elements may also be simplified with
such basis functions. If the added electrons in the bra and ket states are
equivalent, the 
interacting electrons for both one and two-electron operators can all be from
the core states. Such terms only contribute to the diagonal elements, and are
simply the energy values of the core states. When one interacting electron can
be chosen to be the spectator, the kinetic and nuclear potential contributions
are also diagonal, which are their expectation values in the spectator
wavefunctions. With the aid of Racah algebra, the two-electron Coulomb
interactions in such cases may be written as \citep{barshalom88} 
\begin{eqnarray}
<\psi_a,\alpha;J_TM_T|&\sum_{i<j}\frac{1}{r_{ij}}&|\psi_b,\gamma;J_TM_T> = 
\sum_k
(-1)^{j_\alpha+J_b+J_T}\sixj{J_a}{j_\alpha}{J_T}{j_\gamma}{J_b}{k}\nonumber\\
&\times&
\sum_{\beta\delta}<\psi_a||Z^k(\beta,\delta)||\psi_b>
P^k(\alpha\gamma;\beta\delta),
\end{eqnarray}
where $\alpha$ and $\gamma$ are the spectator orbitals, $\beta$ and $\delta$
are the interacting electrons from the core states $\psi_a$ and $\psi_b$, $J_T$
and $M_T$ are the total angular momentum of the new basis function and its
projection, 
the radial integrals $P^k$ are identical to those involved in the DW collision
strength of EIE with free electrons now replaced with the spectators
\citep{gu01b}. 

\subsection{Multipole Radial Integrals}
The radial integrals $M^L_{\kappa\beta}$ are the same as those involved in the
the calculation of bound-bound radiative transitions, with a free electron
replacing one of the bound orbitals. They may be calculated with the fully
relativistic expressions of \citet{grant74}. These expressions depend on
both the photon and the ejected photo-electron energies. However, the current
implementation only takes into account the photo-electron energy dependence,
which is valid when the photon energy is not too high. In fact, when the
photo-electron energy is less than the electron rest mass, which is high
enough for most applications, the non-relativistic limits of the multipole
operators suffices. Therefore, using non-relativistic multipole operators to
evaluate radial integrals is the default of the program. 
Bound and continuum wavefunctions are obtained by solving the Dirac equations,
as discussed in \citetalias{gu01a} and \citetalias{gu01b} in detail.   

Unlike in the bound-bound radiative transition, the radial integral now
depends on the photo-electron energy. In order to reduce the computing time,
when cross sections at a large number of collision energy points are needed,
the $M^L_{\kappa\beta}$ are calculated on an internal grid for the
photo-electron 
energy. The integrals needed at other energies are interpolated on this
grid. The grid points may be specified from the input, or if not specified, it
is generated according to the photo-electron energies where the cross sections
are to be calculated. For highly charged ions, where no Cooper minimum is
present, few points are needed for accurate interpolation. However, when the
PI cross sections show Cooper minima in the low energy region, a dense grid
should be used. On the other hand, the DW approximation at such low energies
in neutral or near neutral ions is rather inaccurate anyway \citep{zhang98}.

\section{User Interface}
\label{sec_program}
The general features of the FAC user interface, and that related to
the calculation of 
atomic structure and EIE collision strengths are presented in
\citetalias{gu01a} 
and \citetalias{gu01b}. In this section, we only discuss routines relevant to
the calculation of PI and RR cross sections. 

When solving the atomic structure problem in preparation for the calculation
of PI and RR cross sections, 
one should use the routine ``SetScreening'' before calling
``OptimizeRadial'' \citepalias{gu01b} to setup the appropriate mean
configuration, if the arguments given to ``OptimizeRadial'' are the
configurations of the recombining ion. Alternatively, one may use
``config.avgconfig(s)'' as 
discussed in \citetalias{gu01a} to specify the mean configuration in
detial. Once the optimized central potential is obtained, and the 
Hamiltonian of the recombining ion is diagonalized, the energy levels for the
recombined ion may be solved in two methods. One is to specify the recombined
configurations, and diagonalize the Hamiltonian as for the recombining ion. An
alternative approach is to use the routine ``RecStates(n, groups)'', where
``n'' is an 
integer specifying the principal quantum number of the spectator electron, and
``groups'' is the configuration group names for the recombining ion. The
basis states will be constructed by coupling the added spectator to the core
states within these configuration groups. Three optional routines,
``SetRecSpectator(n1, n2)'', ``SetRecPWOptions(lmax)'', and
``SetRecPWLimits(m1, m2)'', may be called to control the behavior of the
coupling. The first integer argument, ``n1'',  of ``SetRecSpectator''
specifies the minimum principal quantum number of the spectator electron. If
the ``n'' given to ``RecStates'' is less than ``n1'', the recombined basis
states will be constructed in the normal way, instead of coupling the
spectator to the core states. When the principal quantum number of the
spectator ``n $\ge$ n2'', the mixing between the basis functions having
different core 
states is neglected. ``n2'' must be $\ge$ ``n1'', and may be omitted which
defauts to ``n2 = n1''. If this routine is not called, ``n1 = n2 = 6'' is
used. The routine ``SetRecPWOptions(lmax)'' sets the maximum orbital angular
momentum of the spectator electron to be included. For very high lying
spectator electrons, usually only small orbital angular momentum states
contribute to PI and RR cross sections appreciably. If it is not called, the
default maximum is 12. ``SetRecPWLimits(m1, m2)'' is an alternative way of
specifying the orbital angular momentum of the 
spectator electron to be included, which is the range from ``m1'' to ``m2''
inclusive. 

Once the energy levels of the recombining and recombined ions are
solved, one may optionally specify the internal grid of the photo-electron
energy by calling ``SetPEGrid(n, emin, emax)''. This routime  constructs a
logarithmic 
grid from ``emin'' to ``emax'' with ``n'' points. Alternatively, a Python list
may be given to this routine to specify the grid points in detail. If it is
not called, a 5-point internal grid is generated from the energy points where
the cross sections are to be calculated. These energy points are specified by
the routine ``SetUsrPEGrid'' with the same syntax as ``SetPEGrid''. If the
routine ``SetUsrPEGridType(type)'' is called with
``type = 0'' prior to calling ``SetUsrPEGrid'', the energies specified are
those of the incident photons. If it is called with ``type = 1'', the energies
are those of the photo-electrons, which is the default behavior. However, the
internal energy grid on which the radial integrals are calculated is always
for photo-electrons, the PI and RR cross sections at given photon energies are
obtained by interpolation. The units of energy in these routines is eV.   

The PI and RR cross sections are calculated with the routine ``RRTable(b, f,
fname, m)'', where ``b'' is a list containing the recombined states, ``f'' is
a list containing the recombining
states, ``fname'' is the output file name, and ``m'' indicates the multipole
operator inducing the transition as discussed in the calculation of
bound-bound transitions. If omitted, m defaults to $-1$. ``b'' and ``f'' may be
lists of configuration groups or actual level indices output by FAC. If the
recombined states are constructed by coupling the spectator to the core
states, ``b'' is specified with the Python tuple ``(groups, n)'', where
``groups'' are the configuration groups of the recombining ion and ``n'' is
the principal quantum number of the spectator electron. The results are output
in the file ``fname'' in plain ASCII format. For each transition, the values of
the PI and RR cross sections multiplied by the statistical weight of their
respective initial states are tabulated for the photo-electron energies
specified. The cross sections are in the units of $10^{-20}$ cm$^2$. 

\section{Example}
\label{sec_example}
In this section, a simple example of calculating the PI cross sections of
``2s'' and ``2p'' electrons for the Li-like iron is presented. The Python
script below is the specification of input.
\begin{verbatim}
# import the modules
from pfac import config, fac

fac.SetAtom('Fe')

# specify the configurations for both recombining 
# and recombined ions.
config.config('1s2', group = 'n1')
config.config('1s1 2*1', group = 'n2')
config.config('1s2 2*1', group = 'rn2')

# since the recombined electron is in n=2 shell, 
# set the appropriate screening
fac.SetScreening([2])

fac.OptimizeRadial('n1')

# configuration interaction between n=1 and n=2 
# complexes are included for the recombining ion.
fac.Structure(['n1', 'n2'])
fac.Structure('rn2')

fac.LevelTable('li.lev')

# photo-electron energy grid
fac.SetUsrPEGrid([20.0, 400.0, 2E3, 4E3, 6E3, 10E3])

fac.RRTable(['rn2'], ['n1'], 'li.rr')
\end{verbatim}

Table \ref{tab_comparison} shows the results from the present code and
as compared with those of
\citet{zhang98}. The photo-electron energy grid used in the FAC calculation
is lightly different from that used by \citet{zhang98}. Our results shown in
the table are the 
interpolated values at their energy grid. The ionization potentials and PI
cross sections for the $1s^22s$$\rightarrow$$1s^2$, and
$1s^22p_{1/2,3/2}$$\rightarrow$$1s^2$ transitions are listed. The agreement is
generally good to within a few percent.

\begin{table}
\caption{\label{tab_comparison}}
Comparison of ionization energies (eV) and PI cross sections (Mb) of
$1s^22s$$\rightarrow$$1s^2$, and $1s^22p_{1/2,3/2}$$\rightarrow$$1s^2$ for
Li-like iron. The energy grid is indicated by the photon energy in threshold
units (X).  The first entries are the present results interpolated at the
energy grid used by \citet{zhang98}, and the second entries
are the results of \citet{zhang98}

\begin{tabular}{*{8}{c}}
\hline\hline
&$E_{th}$&1.01&1.20&2.00&3.00&4.00&6.00\\
\hline
$2s$&2046.5&2.35E-2&1.65E-2&5.47E-3&2.13E-3&1.05E-3&3.67E-4\\
$2s$&N/A&2.28E-2&1.62E-2&5.44E-3&2.13E-3&1.06E-3&3.74E-4\\
$2p_{1/2}$&1997.6&2.37E-2&1.42E-2&2.89E-3&7.74E-4&2.93E-4&7.12E-5\\
$2p_{1/2}$&2000.5&2.24E-2&1.36E-2&2.84E-3&7.65E-4&2.91E-4&7.06E-5\\
$2p_{3/2}$&1981.1&2.36E-2&1.40E-2&2.84E-3&7.54E-4&2.84E-4&6.82E-5\\
$2p_{3/2}$&1984.5&2.23E-2&1.35E-2&2.79E-3&7.43E-4&2.81E-4&6.77E-6\\
\hline\hline
\end{tabular}
\end{table}

\section{Conclusions}
A relativistic DW code for calculating non-resonant photoionization and
radiative recombination cross sections is presented. This code is part of an
integrated atomic software package, FAC. A factorized formula for the PI cross
sections and an efficient interpolation procedure for the calculation of
radial integrals are used. The 
test results for Li-like iron are in good agreement with previous works using
similar approximations.  

\section*{Acknowlegements}
The author would like to thank M. Sako and A. Kinkhabwala for their continuous
test of the code, and Peter Beiersdorfer for his helpful suggestions. 
This work is supported by
NASA through Chandra Postdoctoral Fellowship Award Number PF01-10014 issued by
the Chandra X-ray Observatory Center, which is operated by Smithsonian
Astrophysical Observatory for and on behalf of NASA under contract NAS8-39073.

\bibliographystyle{unsrtnat}
\bibliography{facref}

\end{document}
