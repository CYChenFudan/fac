\documentclass[12pt]{article}

\usepackage[body={6.5in,9.5in}]{geometry}
\setlength{\parindent}{0em}

\begin{document}
\title{FAC User Guide}
\author{M. F. Gu}
\maketitle

\begin{abstract}
FAC is a collection of library routines and a Python interface to it for
calculating various atomic processes, including atomic structure, electron
collisional excitation and ionization, photoionization, autoionization, and
their inverse 
processes, radiative and dielectronic recombination. The theoretical method
for the atomic structure is based on a self-consistent Dirac-Fork-Slater
procedure. The excitation, ionization, and recombination implements the
efficient factorization-interpolation method. The Python interface provides
functions for accessing the library from the scripting language Python. This
guide documents these routines in the order of their category.
\end{abstract}

\section{General}
Some functions are general to all atomic processes such as specifying an
electronic configuration. They are collected in this section. In this and the
following sections, if a function name is not fully qualified with its module
name, it is assumed that it comes from module ``fac'', which is the main module
interfacing with the internal ``C'' library. As a convention, all functions
from ``fac'' has names with each word capitalized in first letter.

\begin{description}
\item[SetAtom(s, z, mass)] 
Set the Element. ``s'' is the element symbol, ``z'' is
the atomic number, ``mass'' is the mass of the nucleus. ``z'' and ``mass'' may
be omitted, in which case, their values are determined by looking up an
internal table. e.g., SetAtom('Fe'), SetAtom('Fe', 26), SetAtom('Fe', 26, 56).

\item[config.closed(s)] 
Specify any fully occupied shell for all the
configurations to come. ``s'' is a string denotes an electronic shell in
spectroscopic notation. It may be relativistic or non-relativistic. e.g., 
``1s'', ``2p-'', ``3d'', et al. A ``-'' sign indicates that $j = l - 1/2$. 
An ``*'' in the place of orbital angular momentum symbol denotes any possible
value. This routine may be called several times to specify more than one closed
shell. When it is called with an empty string, it resets to no closed shells. 

\item[config.config(s, group = g)] 
Specify electronic configuration which
belongs to a group with name ``g''. ``s'' is a string denoting a
configuration. Each configuration consists a sequence of shells with
occupation number indicated immediately after the shell. For each shell, the
pricipile quantum 
number comes first, followed by the orbital angular momentum $l$. A ``*'' may
be used to specify $l$ instead of the detailed spectroscopic notation, which
means $l$ may assume any allowed value. After $l$,  A $\pm$ sign
indicates the shell is relativistic with $j$ = $l\pm 1/2$. It may be absent,
which means $j$ may take either value. The occupation number after $j$ may
also be absent assuming a single electron. For any given configuration, the
closed shells specified before are added in front of it. e.g, ``1s2 2*3'', 
``2s1 2p3 3*1'', et al. Multiple configurations belonging to the same group
may be specified in a single call. The Keyword argument ``group = g'' specify
the group these configurations belong to. If it is absent, 
then the last specified group name is used. If none has been specified, they
belong to a single unnamed group. 

\item[AddConfig(gname, cfg)] 
This is the low level routine adding a
configuration to the internal array, and perform angular momentum coupling to
generate basis states. generally, the two functions discussed above should be
used instead of this one. ``gname'' is a string denotes the group name, 
``cfg'' is the numerical representation of a configuration. It must be a
Pythong List with 
each element indicating a relativistic shell. each element is in turn a 4
elements Python Tuple $(n, l, j, nq)$, where $n$ is the pinciple quantum
number, $l$ the orbital angular momentum, $j$ is $\pm 1$ indicates whether it
is $l\pm 1$. $nq$ is the occupation number.

\end{description}

The following two routines set two global options. The default values are
usually sufficient. There should be no need to call them in general. 
\begin{description}
\item[SetMaxRank(k)] 
Set the maximum rank to be included in the multipole
expansion of the coulomb interaction. The default is 6, which is usually
sufficient. 

\item[SetAngZOptions(n, eps1, eps2)] 
Set the options controling the
calculation of angular coefficients. ``n'' is the maximum principal quantum
number beyond which all transitions involving that electron is ignored. If
``n'' is 0, it is effectively infinity, which is the default. 
``eps1'' and ``eps2'' are two threshold numbers. when
the mixing coefficient of a component is smaller than ``eps1'', its
contribution to the angular coefficients is neglected. When the angular
coefficient for specific mixing components is smaller than ``eps2'', its
contribution to the total coefficient is also neglected. The default for
``eps1'' is 1E-3, and for ``eps2'', it is 1E-4. 

\end{description}

The routines in the following group calculates Wigner $nj$ symbols. They are
provided only for demonstration purposes. e.g., they can be used to generate a
table of these coefficients.
\begin{description}
\item[Get3j(j1, j2, j3, m1, m2, m3)] This returns a 3j symbol with given
arguments. All angular momentum parameters are twice their actuall values to
represent the half integer using integers. This is the internal
representation also. 

\item[Get6j(j1, j2, j3, j4, j5, j6)] Returns a 6j symbol.

\item[Get9j(j1, j2, j3, j4, j5, j6, j7, j8,j9)] Returns 9j Symbol.

\item[GetCG(j1, j2, j3, m1, m2, m3)] Returns the Clebsch-Gordan coefficients.

\end{description}

A cubic spline interpolation routine is exported from fac, which may be
useful. 
\begin{description}
\item[y2 = Spline(x, y, dy1, dy2)]
Calculates the cubic spline coefficients for given ``x'' and ``y'' arrays,
with the second derivative at both ends ``dy1'' and ``dy2''. If ``dy1'' and
``dy2'' are missing, a natural cubic spline is assumed. The functions returns
a List contains the second derivatives of the spline at each node.

\item[y0 = Splint(x, y, y2, x0)] 
Evaluates the interpolation at ``x0'', ``y2'' is the List returned by
Spline(). 
\end{description}

\section{Atomic Structure}
The atomic structure part calculates the energy levels for the specified
configurations, and the radiative transition rates between the resulting
levels. The calculation is carried out in the suppoposion of configuration
approximation. In this approximation, a set of radial basis is first
constructed. They are then used to set up the Hamilton matrix, the
diagnolization of which gives the energy levels and eigen states. The
construction of radial basis is based on an average configuration
self-consistent field procedue. 

The following routines set some global options controling the computation.
\begin{description}
\item[SetOptimizeControl(tol, maxiter, iprint)] 
Set parameters affecting the optimization of radial central potentials. 
``tol'' is the desired accuracy for self-consistent iterations. ``maxiter'' is
the maximum iteration the optimization may perform. ``iprint'' indicates
whether iteration summary should be printed out. The defaults: ``tol = 1E-6'',
``maxiter = 100'', ``iprint = 0''. 

\item[SetScreening(screen, charge, kl)] 
Set the principle quantum numbers, which should be included in optimizing the
radial potential. This is used to provide additional electron sreening for
recombined bound orbitals and free orbitals. The total charge specified by
``charge'' is distributed among all shells specified by ``screen'', which is
and integer list. For each principal quantum number in ``screen'', if ``kl $<$
0'', the s-orbital is used, if ``kl $>$ 0'', the nodeless orbital is used, if
``kl = 0'', the orbital angular momentum is the principal quantum number
divided by 2, the total angular momentum of the orbital is always
$l+1/2$.  ``charge'' and ``kl'' may be absent, which defaults to 1.0 and a
nodeless orbital. 

\item[SetRadialGrid(rmin, rmax)]
Set the radial grid limits. The actual grid limits are ``rmin/z'', and 
``rmax/z'', where ``z'' is the residual charge of the atomic ion. The
defaults: ``rmin = 1E-5'', ``rmax = 5E3''. The large value for ``rmax'' is
needed for the calculation of very high Rydberg levels. If they are not needed,
a smaller value may be chosen. 

\item[SetTransitionOptions(gauge, mode, maxe, maxm)] Sets options controling
the calculation of transition rates. ``gauge'' specifies the gauge: 
\begin{enumerate}
\item[1.] Coulomb (velocity form).
\item[2.] Babushkin (length form). 
\end{enumerate}
If mode = 0, use relativistic expressions
for the radial integral. If mode = 1, use non-relativistic approximation. 
``maxe'', and ``maxm'' are the maximum rank of multipoles allowed in the
calculation. The default is to use non-relativistic radial expression in
length form. ``maxe = maxm = 4''. 

\item[SetTransitionCut(eps)] 
When the transition rate is less than ``eps'' fraction of the total decay rate
for a given level, it is omitted in the output. Default: ``eps = 1E-4''. 

\end{description}

The following routines are called to perform the calculation.
\begin{description}
\item[OptimizeRadial(groups)] 
Carry out the radial optimization. ``groups'' are a set of configuration
groups on which the optimzation is based. It may be a Python list or just a
sequence of group names as variable length arguments. Internally, a mean
configuration with fractional 
occupation numbers is constructed. A self-consistent Dirac-Fork iteration
is then performed for this mean configuration. A central radial potential is
then determine including the electron-electron interaction (both direct and
exchange) with this mean configuration. The central potential is used to
construct the radial basis for configuration interaction. 

\item[Structure(groups, perturbing\_groups)] 
Setup the Hamilton matrix, and diagonalize
it. Only the configurations in ``groups'' and ``perturbing\_groups'' are
assumed to interact. However, configurations in ``perturbing\_groups'' are of
minor importance. The levels belong to them are not calculated. They exist
only to interact with ``groups''. In setting up the Hamiltonian, only the
diagonal matrix elements within ``perturbing\_groups'' and the non-diagonal
elements between ``groups'' and ``perturbing\_groups'' are retained. If no
argument is given, all configurations are mixed.  

\item[LevelTable(fname, n, m)] 
Output the Energy Level to an ascii file ``fname''. ``n'' is the principle
quantum number, only levels with the last orbital having such ``n'' is printed
out. If ``n == 0'', all levels are output. ``m'' indicates whether the mixing
coefficients should be printed out. The last two arguments may be absent which
defaults to ``n = 0'', ``m = 0''.

\item[TransitionTable(low, up, fname, multipole)] 
Calculate the transition rates between lower and upper levels specified by
``low'' and ``up''. Each one may be a list of configuration group names or a
list of level indices. ``fname'' is the output file name. ``multipole''
specifies the mutipole of the transition desired, negative number means
electric multipole transition, and positive number means magnetic multipoles,
the absolute value is the rank of the multipole, which must be less than or
equal to the maximum ranks set before. 
\end{description}


\section{Collisional Excitation}
This part calculates the electron direct collisional excitation cross
sections in the distorted wave approximation. In the following routines, all
appearence of energy has the unit of eV, although internally, the code uses
the atomic unit.

\begin{description}
\item[SetCEPWOptions(max\_allow, max\_forbidden, qr, tol)] 
Set partial wave options for excitation calculation. ``max\_allow'' is the
maximum partial wave to be included for dipole allowed transitions. 
``max\_forbidden'' is that for forbidden transitions,
Internally, a partial wave grid is setup from $l = 0$ to 
``max\_allow'', which includes every $l$ below 5 but use a logarithmic grid
beyond that. The radial integrals are then interpolated on this grid. If
a detailed specification of the grid is needed, SetCEPWGrid() may be used.
``qr'' is the orbital
angular momentum value beyond which a ``quasi-relativistic'' approximaiton
is used for the radial integrals, which ignores the small component of the
wavefunction and treat the orbitals with different $j$ but same $l$ as the
same. ``tol'' is a small number, When the exchange integral is smaller than the
tol fraction of the direct integral, the exchange integral is neglected, and a
separated form of the Coulomb interaction is used. The defaults: ``max\_allow
= 40'', ``max\_forbidden = 15'', ``qr = 1'', ``tol = 1E-3''.

\item[SetCEPWGrid(m)] 
``m'' is the partial wave value, beyond which a logrithmic
grid is used. This function has a more detailed form, taking two equal length
python lists as arguments. e.g., if the two lists are [5, 2], and [1, 2], then
the partial wave grid generated will be [0, 1, 2, 3, 4, 5, 7, 9, 13, 17,...].
ie., the numbers in the 1st list is the repetition number for the increment
step in the 2nd list. when the list is exhausted, the grid continues, by
doubling the increment every two points until it reaches the maximum partial
wave specified by SetCEPWOptions(). Default ``m = 5''

\item[SetTEGrid(n, emin, emax)] 
Set the transition energy grid used in the
interpolation of radial integrals. Usually, it does not need to be
called. for a given transition array, the transition energy grid is set
automatically, with 3 points, which is sufficiently accurate for the
interpolation of the entire transition array. 

\item[SetCEGrid(n, emin, emax)] 
Set the scattered electron energy grid. From emin to emax with n points in a
logrithmic grid. This function may also take a list of energy points. This
grid is used internally to calculate the radial integrals. The output energy
grid is specified with SetUsrCEGrid(). Usually, the internal grid is generated
automatically according to the user output grid. 

\item[SetUsrCEGrid(n, emin, emax)]
Set the output grid for the scattered electron energy. Arguments are the same
as SetCEGrid().

\item[CETable(low, up, fname)] Calculates the collision strength of the
transition array specified by ``low'' and ``up''. Output the results to file
``fname''. ``low'' and ``up'' may be group names or actual level indices.

\item[CETableMSub(low, up, fname)] Calculates the collision strength from and
to the specific magnetic sublevels of the ``low'' and ``up'' levels.

\end{description}

\section{Ionization}
The calculation of collisional ionization is similar to that of excitation, we
refer to the parallel routines in the excitation for the syntax.
\begin{description}
\item[SetCIPWOptions(max\_a, max\_f, max\_k, max\_e, qr, tol)] 
See SetCEPWOptions() for detail. Additional parameters ``max\_k'' and
``max\_e'' are optional, specifies the maximum rank of Coulomb expansion for
ionization radial integrals (if a smaller value than that specified by
``SetMaxRank'' is desired'' and the maximum orbital angular momentum of the
ejected electron. Defaults are ``max\_a = 15'', ``max\_f=10'', ``max\_k=6'',
``max\_e=6'', ``qr=0'', ``tol=1E-3''.

\item[SetCIPWGrid(m)]
See SetCEPWGrid() for detail.

\item[SetIEGrid(n, emin, emax)]
See SetTEGrid(). This sets the ionization threshold energy grid, int the same
way as the excitation energy grid.

\item[SetCIEGrid(n, emin, emax)]
See SetCEGrid().

\item[SetUsrCIEGrid(n, emin, emax)]
See SetUsrCIEGrid(). The energy grid specified is the sum of two outgoing
electrons, i.e., E0-I, where E0 is the energy of the impacting electron, and I
is the ionization threshold.

\item[CITable(bound, free, fname)] 
Calculate the collisional ionization cross sections between the ``bound''
states and ``free'' states, output the results in ``fname''.
\end{description}

\section{Recombination}
This part calculates photoionization, autoionization rates, and their inverse
processes, radiative and dielectronic recombination.
\begin{description}
\item[SetRecSpectator(n1, n2)] 
Set the recombined electron principal quantum
number options. Above ``n1'', one constructs the recombined states in terms of
the exsisting target states. In this case, all the electrons in the target
must have $n < n1$. Above ``n2'', The interaction between basis having
different target states is ignored. Usually, setting $n1 = n2$ to a suitablely
high value (or simply giving one argument). The defaults: ``n1 = n2 = 6''.

\item[SetRecPWOptions(max)] 
Set the maximum partial waves included in the
calculation. max = 6 - 8 is usually sufficient. For DR with the transition in
the core, however, max should be around 12. The default is 12.

\item[SetRecPWLimits(m1, m2)] 
Further limit the orbital angular momentum value of the recombined electron
in the range ``m1'' and ``m2'' inclusive.

\item[RecStates(n, groups)] 
Construct recombined states based on the  target
states from ``groups'', and captured electron in ``n'' levels. The recombined
states constructed this way can be referred to by a tuple (groups, n). 

\item[SetPEGrid(n, emin, emax)] 
Set the internal photo-electron energy grid in photo-ionization and radiative
recombination calculations. A Python list may also be used to specify the
detailed energy grid. 

\item[SetUsrPEGrid(n, emin, emax)]
Set the photo-electron energy grid where photo-ionization and radiative
recombination cross sections are desired. The radial integrals involved will
be interpolated from the internal grid set by SetPEGrid.

\item[RRTable(bound, free, fname, m)] Calculates the photoionization cross
section and radiative recombination cross section between ``bound'' and
``free'' states. Output results to file ``fname''. ``bound'' and ``free'' can
be specified by actual level indices or by the group names. ``m'' specifies
the multipole type of the transition. It may be ommited, which defaults to E1
transtions, ``m = -1''.

\item[SetAICut(eps)] When the autoionization rate is less than ``eps''
fraction of the total AI rate for that level, it is omitted in the output. The
default is 1E-8.

\item[SetDRTEGrid(n, emin, emax)] 
Set the continuum electron energy grid in
autoionization calculation. The radial integrals needed are interpolated on
this grid. It does not need to be called explicitly. For given transition
arrays, they are set automatically. 

\item[CorrectEnergy(levels, energies)]
Corrects the energy value of ``levels'' to ``energies'', where ``levels'' and
``energies'' are Python lists. the energies specified are relative to the
ground state, and in the unit of eV. Instead of giving two lists as arguments,
one may also give the name of file, which contains two columns, one for level
indices, and the other for energy values. This is useful to correct the energy
of low energy resonances according to experimental values. However, this only
works if the recombined states are constructed using ``RecStates'', which is
often the case for low energy resonances. 

\item[nopen=DROpen(groups)]
Calculates the principal quantum numbers where DR channels corresponds to the
levels in ``groups'' start to open according to the hydrogenic formula. The
returned list is used to construct the 
principal quantum number grid in the DR calculation. 

\item[AITable(bound, free, fname)] Calculates the autoionization width from
``bound'' to ``free'' states. 

\item[DRTable(free, ai, bound, g, aifile, trfile)] This combines the
calculation of autoionization rates and radiative transition rates in a single
routine. All autoionization channels between ``ai'' and ``free'' are
included. ``g'' specifies the levels to which the DR recombination is to be
computed. The radiative transition rates between ``ai'' and ``bound'' are
calculated if ``ai'' is accessable from ``g''. The resulting autoionization
rates and transition rates are output in ``aifile'' and ``trfile''
respectively. 

\end{description}

There is a python module ``dr.py'' that makes the calculation of DR easier. 
\begin{description}
\item[dr.dr()] 
It accepts a long parameter list (prefix, groups, reclevel, ground, aigroup,
decaygroup, channels, ngrid, nmin, nj0, maxkl, nspec, correlations). 
This routine coordinates the DR calculation for various DR
channels. ``prefix'' is a string which is used to construct the output file
names. ``gr'' is a list of target configuration groups that are
involved. ``reclevel'' is a list of levels from which the recombination is to
ocuur, e.g., [0], if only DR onto the ground level is to be
calculated. ``ground'' is the target state group to which autoionization is
allowed. ``aigroup'' is the bound autoionizing groups. ``decaygroup'' is the
bound group to which the radiative decay is allowed. ``channel'' gives a
numerical symbol to each DR channel. ``ngrid'' is the a grid of principal
quantum numbers for the recombined electron. Detiled calculation is only
performed for these levels. Other missing levels are obtained through
interpolation. ``nmin'' is the minimum principal quantum number in the
construction of recombined states. ``nj0'' is the maximum principal quantum
number to which the decay of autoionizing levels are allowed. ``nspec'' is the
option to be used in SetRecSpectator(). ``correlations'' regroups the list of
target configurations to specify the configuration mixing schemes. See a
sample script under the directory ``demo'' to see how it is used.

\item[dr.drall(prefix, nrec, c, rate=rate, cross=cross, strength=s, drs=drs)]
This routine proccess the output from the DR calculation, calculates the rate
coefficients, cross sections, DR strengthes, and DR satellites. ``prefix'' is
the file name 
prefix. ``nrec'' is the principal quantum number grid used for channel
``c''. ``rate = [Tmin, Tmax, n]'' specifies the temperature grid for the rate
coefficients. ``cross = [width, vmultiply]'', ``width'' is the broadening of
the DR resonances. ``vmultiply'' indicates whether the velocity factor should
be multipled to the cross section to produce the rate
coefficient. ``strength=1'' specify whether the DR strength for each
recombined complex should be calculated. ``drs=[lambda0, lambda1, delta]''
specify the wavelength limits and spectral resolution in the calculation of DR
satellite lines. 

\item[dr.sumrate(outfile, prefix, nrec, nopen, c, nmax=nmax)]
This routine sums up the rate coefficients for all recombined complexes. The
interpolation on the ``n'' grid is carried out. From the last calculated ``n''
to nmax, extrapolation is used with the $n^{-3}$ scaling law.
\end{description}

\end{document}

